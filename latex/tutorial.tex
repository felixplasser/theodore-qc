\documentclass[DIV=12,headings=normal]{scrartcl}

\usepackage[utf8]{inputenc}
\usepackage[T1]{fontenc}
\usepackage[ngerman,USenglish]{babel}

\usepackage[intlimits]{amsmath}
\usepackage{amssymb}
\usepackage{icomma}
\usepackage{braket}
\usepackage{color}
\usepackage{listings}
\usepackage{hyperref}

\usepackage{graphicx}
%\usepackage{subfig}
\usepackage{booktabs}
\usepackage{pdflscape}
\usepackage{subfigure}

\usepackage{tikz}
% \usetikzlibrary{decorations}
% \usetikzlibrary{decorations.pathmorphing}
% \usetikzlibrary{plotmarks}
% \usetikzlibrary{patterns}
\usepackage{pgfplots}
\pgfplotsset{compat=1.3}
\usepgfplotslibrary{groupplots}

\usepackage{siunitx}
\sisetup{decimalsymbol=comma}
\usepackage{multirow}
\usepackage{url}
\usepackage{fancyhdr}

\usepackage[version=3,arrows=pgf]{mhchem}

\usepackage{lmodern}
\usepackage{fancyvrb}

\setkomafont{subject}{\usekomafont{title}}
\setkomafont{caption}{\small}
% \captionsetup[subfloat]{font={default,small}}

\newsavebox\MBox
\newcommand\Cline[2][red]{{\sbox\MBox{#2}%
  \rlap{\usebox\MBox}\color{#1}\rule[-1.2\dp\MBox]{\wd\MBox}{0.5pt}}}

\newcommand{\comment}[1]{\textcolor{blue}{#1}}
\newcommand{\red}[1]{\textcolor{red}{#1}}
\newcommand{\redl}[1]{\Cline{\textcolor{red}{#1}}}
\newcommand{\reds}[1]{\redl{\scriptsize #1}}

\newcommand{\comm}[1]{
\small
%\begin{Verbatim}%[commandchars=\\\{\}]
~> \redl{#1}
%\end{Verbatim}
\normalsize
}
\newcommand{\incmo}[1]{\includegraphics[trim=1cm 1cm 1cm 1cm, clip=true, width=4cm]{#1}}

\let\origttfamily=\ttfamily % alte Definition von \ttfamily sichern
\renewcommand{\ttfamily}{\origttfamily \hyphenchar\font=`\-}

\newcommand{\greybox}[1]{
  \vspace{3mm}
  \fcolorbox{black}{black!15}{
    \begin{minipage}{0.9\textwidth}\textit{#1}\end{minipage}}
  \vspace{3mm}
}

\newcommand{\todo}[1]{\textcolor{red}{#1}}
\newcommand{\col}{\textsc{Columbus}}
\newcommand{\molcas}{\textsc{Molcas}}
\newcommand{\seward}{\texttt{Seward}}

\newcommand{\theo}{\textsc{TheoDORE}}

\newcounter{number}
\newcommand{\numbering}[1]{\thenumber. #1\addtocounter{number}{1}}
\newcommand{\renumber}{\setcounter{number}{1}}

\pgfplotsset{every tick label/.append style={/pgf/number format/use comma,/pgf/number format/1000 sep={}}}
\pgfplotsset{scaled ticks=false}
\pgfplotsset{small,width=0.6\textwidth,max space between ticks=35}
\pgfplotsset{tick style={thin,black}}

%%%%%%%%%%%%%%%%%%%%%%%%%%%%%%%%%%%%%%%%%%%%%%%%%%%%%%%%%%%%%%%%%%%%%%%%%%%%%%%%%%%%%%%

\fancyhf{}				%Leeren aller Header, Footer
\fancyhead{}		%Links
\fancyfoot[L]{\theo{} tutorial}
\fancyfoot[R]{\thepage}			%Seite x von y
\renewcommand{\headrulewidth}{0pt}	%Linie unter Header

%%%%%%%%%%%%%%%%%%%%%%%%%%%%%%%%%%%%%%%%%%%%%%%%%%%%%%%%%%%%%%%%%%%%%%%%%%%%%%%%%%%%%%%

%\subject{\Large A Tutorial for \theo}
\title{\LARGE A Tutorial for \theo~1.2}
\author{\large Felix~Plasser}
\publishers{\small \textit{Institute for Theoretical Chemistry -- University of Vienna}}
\date{\large Vienna, 2016\\[3em]}

\begin{document}

\pagestyle{fancy}
\selectlanguage{USenglish}

\maketitle
%\cleardoublepage

\tableofcontents
\clearpage

\section{Before Starting}

\subsection{Introduction}

This tutorial will give you an introduction into various functionalities of the \theo{} program package.

\subsection{Installation}

For the installation, follow the steps described at

\url{https://sourceforge.net/p/theodore-qc/wiki/Installation/} \\

It is necessary to set up, both, the \texttt{PATH} and \texttt{PYTHONPATH} environment variables.

\subsection{Notation}

The following notation is used:

\scriptsize
\begin{Verbatim}[commandchars=\\\{\}]
This kind of font indiates what is seen on the screen
\redl{and the command lines that you should write <ENTER>}  \comment{! Comments come here}
\end{Verbatim}
\normalsize

\greybox{Important information related to \theo{} but not necessarily connected to the current job comes in boxes like this.}
\clearpage
\section{Natural transition orbitals}

As a first step, we will plot the natural transition orbitals (NTOs) in the case of the formaldehyde dimer computed with RI-CC2 in \textsc{Turbomole}.

\subsection{Input generation}
\label{sec:inpnto}

The input files are taken from the \texttt{EXAMPLES} directory in the \theo{} distribution

\comm{cp -r \$THEODIR/EXAMPLES/fa2.ricc2/QC\_FILES/ Tutorial} \\

Inside the \texttt{Tutorial} directory, run the input program

\comm{theoinp}

\scriptsize
\begin{Verbatim}[commandchars=\\\{\}]
Type of job (rtype):
  [ 1]      qcadc - Q-Chem ADC (libwfa output)
  [ 2]     libwfa - General libwfa output
  [ 3]    qctddft - Q-Chem TDDFT
  [ 4]   colmcscf - Columbus MCSCF
  [ 5]    colmrci - Columbus MR-CI (tden analysis)
  [ 6]      rassi - Molcas RASSI
  [ 7]        nos - Read natural orbitals (Molden format) for sden analysis: Columbus, Molcas, ...
  [ 8]      ricc2 - Turbomole ricc2
  [ 9]       escf - Turbomole escf
  [10]      cclib - Use external cclib library: Gaussian, ORCA, GAMESS, ...
Choice: [8] \redl{8 <ENTER>} \comment{! \texttt{theoinp} tries to guess the program used according to the files present}

Main file to read (rfile):
Choice (autocomplete enabled): [ricc2.out] \redl{<ENTER>}

MO file (Molden format)
 -> This file should ideally contain a square invertible coefficient matrix (mo_file):
Choice (autocomplete enabled): [molden.input] \redl{<ENTER>}

 *** Warning: in the case of ricc2 you have to delete the line
       implicit core=   x virt=    x
     from the control file before running tm2molden.
 \comment{! A consistent file is given here. But in general, one should be careful when using \texttt{ricc2}}

Analysis of transition density matrices?
Choice (y/n): [y] \redl{<ENTER>}

Perform CT number analysis?
Choice (y/n): [y] \redl{n <ENTER>} \comment{! Do not perform a charge transfer number analysis to keep things simple}

Perform natural transition orbital (NTO) analysis?
Choice (y/n): [y] \redl{<ENTER>}

NTOs as Jmol script? (jmol_orbitals):
Choice (y/n): [y] \redl{y <ENTER>} \comment{! Type "y" if you have the \textsc{Jmol} program available}

NTOs in Molden format (molden_orbitals):
Choice (y/n): [n] \redl{y <ENTER>} \comment{! Type "y" if you want files in \textsc{Molden} format}

Use alpha/beta rather then negative/positive to code for hole/particle orbitals? (alphabeta):
Choice (y/n): [n] \redl{n <ENTER>} \comment{! Only for special applications}

Perform exciton analysis?
Choice (y/n): [y] \redl{n <ENTER>}

Adjust detailed output options?
Choice (y/n): [n] \redl{<ENTER>}

Name of input file
Choice: [dens_ana.in] \redl{<ENTER>}
Finished: File dens_ana.in written.
\end{Verbatim}
\normalsize

After going through these steps, the file \texttt{dens\_ana.in} with the following content is written:

\scriptsize
\begin{Verbatim}[commandchars=\\\{\}]
rtype='ricc2'
rfile='ricc2.out'
mo_file='molden.input'
comp_ntos=True
jmol_orbitals=True
molden_orbitals=True
alphabeta=False
prop_list=['PRNTO']
\end{Verbatim}
\normalsize

\greybox{To learn more about the available keywords check:\\
\url{https://sourceforge.net/p/theodore-qc/wiki/Keywords/}
}


\subsection{Transition density matrix (1TDM) analysis}

To run the 1TDM analysis to produce the NTOs, simply type:

\comm{analyze\_tden.py}

After some technical information, you will find the following output summary

\scriptsize
\begin{Verbatim}[commandchars=\\\{\}]
state       dE(eV)     f  PRNTO \comment{! label of the state / exc. energy / osc. strength / NTO participation ratio}
-------------------------------
1(1)a        4.174 0.000  1.943
2(1)a        4.192 0.000  1.952
3(1)a        7.944 0.000  1.849
4(1)a        8.021 0.164  1.882
5(1)a        8.755 0.000  1.991
6(1)a        8.763 0.052  1.998
\end{Verbatim}
\normalsize
\clearpage

\subsection{Plotting of the orbitals}

If you selected to export files in \textsc{Molden} format, one file for each individual state will be present

\scriptsize
\begin{Verbatim}[commandchars=\\\{\}]
nto_1-1-a.mld  nto_2-1-a.mld  nto_3-1-a.mld  nto_4-1-a.mld  nto_5-1-a.mld  nto_6-1-a.mld
\end{Verbatim}
\normalsize

You can visualize them with any program of your choice.

In the case of using \textsc{Jmol} a short cut is available.
To plot all the orbitals in one go, simply type

\comm{jmol -n nto\_jmol.spt}

Then \textsc{Jmol} will create \texttt{.png} files for all the orbitals.
To look at all these orbitals at once open the file \texttt{nto.html} in a browser.\\

\begin{tabular}{|ccl|}
\hline
\textbf{1-1-a} &&\comment{! state label}\\
\comment{! hole/occupied NTO} & \comment{! particle/virtual NTO}\\
\incmo{fa2ntos/NTO1-1o-56.png} & \incmo{fa2ntos/NTO1-1v-56.png} &\\
0.556 & 0.556 & \comment{! NTO amplitudes $\lambda_i$}\\
\incmo{fa2ntos/NTO1-2o-39.png} & \incmo{fa2ntos/NTO1-2v-39.png} &\\
0.394 & 0.394&\\
\hline
\textbf{2-1-a}&&\\
\incmo{fa2ntos/NTO2-1o-56.png} & \incmo{fa2ntos/NTO2-1v-56.png} &\\
0.556 & 0.556 &\\
\incmo{fa2ntos/NTO2-2o-41.png} & \incmo{fa2ntos/NTO2-2v-41.png} &\\
0.405 & 0.405&\\
\hline
\end{tabular}

\greybox{In both cases two NTO pairs are required to describe that transition, in agreement with ${\rm PR}_{\rm NTO}\approx 2$.}
\clearpage

\section{Charge transfer number analysis}
In this part an analysis of the charge transfer numbers is carried out.
This requires dividing the system into fragments, which are analyzed together.

\section{Using the external cclib library}

\scriptsize
\begin{Verbatim}[commandchars=\\\{\}]

\end{Verbatim}
\normalsize


\end{document}